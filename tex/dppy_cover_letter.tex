\documentclass[twoside,11pt]{article}

% Any additional packages needed should be included after jmlr2e.
% Note that jmlr2e.sty includes epsfig, amssymb, natbib and graphicx,
% and defines many common macros, such as 'proof' and 'example'.
%
% It also sets the bibliographystyle to plainnat; for more information on
% natbib citation styles, see the natbib documentation, a copy of which
% is archived at http://www.jmlr.org/format/natbib.pdf

% !TEX root = dppy_paper.tex
\usepackage{jmlr2e}
\usepackage{amsfonts, amsmath, amssymb}
\usepackage{fontawesome} % For fancy icons :)
\usepackage{enumitem}
\usepackage[usenames, dvipsnames, svgnames]{xcolor}
\definecolor{mydarkblue}{rgb}{0,0.08,0.45}
\usepackage{hyperref}
\hypersetup{
    breaklinks=true,
    colorlinks=true,
    linkcolor=mydarkblue,
    citecolor=mydarkblue,
    filecolor=mydarkblue,
    urlcolor=mydarkblue
}
\usepackage[utf8x]{inputenc}
\usepackage{pgffor} % To create math commands recursively in commands.tex
\usepackage{natbib}

% Code display
\usepackage{listings}
\usepackage{dblfnote}
\DFNalwaysdouble % for this example

% Specific to this .tex
% DPP
\newcommand{\DPPy}{\textsf{DPPy}}
\newcommand\DPP{\operatorname{DPP}}

% Python code highlight
\newcommand{\pygreen}[1]{\textcolor{OliveGreen}{#1}}
\newcommand{\pystr}[1]{\textcolor{BrickRed}{"#1"}}
\newcommand{\pylrcb}[1]{\string{#1\string}}
\newcommand{\pykwargs}{\textcolor{Plum}{**}}
\newcommand{\pyeq}{\textcolor{Plum}{=}}
\newcommand{\pyus}{\texttt{\char`_}}

% TODO command
\newcommand*{\todo}[1]{\textcolor{red}{TODO:{#1}}}
\newcommand*{\anstodo}[1]{\textcolor{blue}{ANS-TODO:{#1}}}

% Notations
\newcommand\eg{\text{e.g., }}
\newcommand\iid{\text{i.i.d.\,}}

% Footnotes 1, 2, 3 and then fancy ones!
\makeatletter
\@addtoreset{footnote}{page}
\makeatother

\renewcommand{\thefootnote}{\ifcase\value{footnote}\or\color{black}{1}\or\color{black}{2}\or\color{black}{3}\or\color{black}{\faGithub}\or\color{black}{\faBook}\or\color{black}{\faGears}\fi} 

% Cross ref a foonote
\makeatletter
\newcommand\footnoteref[1]{\protected@xdef\@thefnmark{\ref{#1}}\@footnotemark}
\makeatother

\addtolength{\skip\footins}{-3pc plus 5pt}

% Center env for minted package to display code
\newenvironment{nscenter}
 {\parskip=3pt\par\nopagebreak\centering}
 {\parskip=2pt\par\noindent\ignorespacesafterend}

\newcommand*{\Eqref}[1]{Equation~\ref{#1}}

%%%%%%%%%
% MATHS %
%%%%%%%%%

\renewcommand{\tilde}{\widetilde}

% Sum/Product with limits
\newcommand{\suml}{ \sum\limits }
\newcommand{\prodl}{ \prod\limits }

%%% bold, cal and bb Uppercase letters A..Z
\foreach \x in {A,...,Z}
	{%
	\expandafter\xdef\csname bf\x \endcsname{\noexpand\ensuremath{\noexpand\bf{\x}}}
	\expandafter\xdef\csname cal\x \endcsname{\noexpand\ensuremath{\noexpand\mathcal{\x}}}
	\expandafter\xdef\csname bb\x \endcsname{\noexpand\ensuremath{\noexpand\mathbb{\x}}}
}

\newcommand\Vol{\operatorname{Vol}}

% differential notation dx
\newcommand*\diff{\mathop{}\!\mathrm{d}}

%%%%%%%%%%%%%%%%%%%%%%%
% Size adaptive symbols
%%%%%%%%%%%%%%%%%%%%%%%
% brackets
\newcommand{\lrb}[1]{\left[ #1 \right]}
% parenthesis
\newcommand{\lrp}[1]{\left( #1 \right)}
% curly brackets (typically sets)
\newcommand{\lrcb}[1]{\left\{ #1 \right\}}
% absolute value
\newcommand{\lrabs}[1]{\left| #1 \right|}
% norm
\newcommand{\lrnorm}[1]{\left\| #1 \right\|}

%%%%%%%%%%%%%%%%%%%%%
% Probability symbols
%%%%%%%%%%%%%%%%%%%%%

\newcommand\Prob{\operatorname{\bbP}}
\newcommand\Exp{\operatorname{\bbE}}

% Adaptive brackets
\newcommand{\Proba}[1]{\Prob\lrb{#1}}
\newcommand{\Expe}[1]{\Exp\lrb{#1}}

% Independent symbol similar to perp
\def\independenT#1#2{\mathrel{\rlap{$#1#2$}\mkern2mu{#1#2}}}
\newcommand\indep{\protect\mathpalette{\protect\independenT}{\perp}}

% Distributions
\newcommand\Ber{\operatorname{\calB er }}

%%%%%%%%%%%%%%%%
% Linear Algebra
%%%%%%%%%%%%%%%%
\renewcommand{\top}{\mathsf{\scriptscriptstyle T}}

\newcommand\Span{\operatorname{span}}

\newcommand\Tr{\operatorname{Tr}}
\newcommand\rank{\operatorname{rank}}

\setlength\parindent{0pt} % Unindent paragraphs
\setlength{\parskip}{2.5mm}

% Heading arguments are {volume}{year}{pages}{date submitted}{date published}{paper id}{author-full-names}

\jmlrheading{xx}{2019}{xx-xx}{8/12}{xx/xx}{gabava18}{Gautier,
                                                     Guillermo Polito,
                                                     Rémi Bardenet
                                                     and Michal Valko}

% Short headings should be running head and authors last names

\ShortHeadings{\DPPy}{Gautier,
                      Polito,
                      Bardenet
                      and Valko}
\firstpageno{1}

\begin{document}

\title{\DPPy: Sampling DPPs with Python\\[1pt]
\textit{\normalsize Cover letter}}

\author{\name Guillaume Gautier
            \email g.gautier@inria.fr\\
        \name Guillermo Polito
            \email guillermo.polito@univ-lille.fr\\
        \name R\'emi Bardenet
            \email remi.bardenet@gmail.com\\
        \name Michal Valko
            \email valkom@deepmind.com
      }

\editor{}

\maketitle

% \title{\DPPy: Sampling DPPs with Python\\[5pt]
% \textit{\normalsize Cover letter}}

% \author{\name Guillaume Gautier \email g.gautier@inria.fr \\
%        \name R\'emi Bardenet \\
%        \name Michal Valko
% }

% \maketitle

\vspace{-4em}

\textbf{Keywords:}
% \begin{keywords}%
    determinantal point processes,
    sampling,
    MCMC,
    random matrices,
    Python
% \end{keywords}

\vspace{1em}

\pagenumbering{gobble} % Remove page numbering
\setcounter{footnote}{3}

Dear Editorial Board,\\

We submit the \DPPy\ Python library to the machine learning open source software (MLOSS) section of the Journal of Machine Learning Research.
\DPPy\ is the acronym for Determinantal Point Processes (DPPs) with Python.

\DPPy\ tackles the challenging task of sampling DPPs with a turnkey implementation of a wide diversity of exact and approximate algorithms known so far to sample finite DPPs.
We also include algorithms for non-stationary continuous DPPs, e.g., related to random covariance matrices or Monte Carlo methods that are also of interest for MLers.

The \DPPy\ project is hosted on GitHub\!\footnote{\footGitHubDPPy}and supported by an extensive documentation\!\footnote{\footReadTheDocs} which provides the essential mathematical background and illustrates the key properties of DPPs through \DPPy\ objects and their methods.
\setcounter{footnote}{5}Travis\!\footnote{\footTravis}is used for continuous integration to ensure that \DPPy\ runs at least on Linux with Python 3.6+.
\textbf{Since the last submission, the coverage rate has increased from 14\% to 90\%, see the Coveralls\!\footnote{\footCoveralls}page}.

For better reproducibility, \DPPy\ does not rely on any proprietary software and is released under the MIT license.
Moreover, the companion paper\!\footnote{\footGitHubDPPyPaper}together with the present cover letter are also available on GitHub and fully reproducible.

We declare that this work is neither published nor submitted to any journal or conference.
All authors agree to the reviewing process of the Journal of Machine Learning Research.

Thank you for your consideration.

\textbf{Version to be reviewed}: \DPPy\ \href{https://github.com/guilgautier/DPPy/releases/tag/v0.2.0}{v0.2.0}

\end{document}
