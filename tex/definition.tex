% !TEX root = dppy_paper.tex
\subsection{Definition} % (fold)
\label{sub:definition}

A point process $\calX$ on $\bbX$ is a random subset of points $\lrcb{X_1, \dots, X_N} \subset \bbX$, where the number of points $N$ is itself random.
We further add to the definition that $N$ should be almost surely finite and that all points in a sample are distinct.
Given a reference measure $\mu$ on~$\bbX$, a point process is usually characterized by its $k$-correlation function $\rho_k$ for all $k$, where
\begin{equation*}
\label{eq:correlation_function_intuition}
	\Proba{
		\begin{tabular}{c}
			$\exists$ one point of the process in\\
			each ball $B(x_i, \diff x_i), \forall i=1,\dots, k $
		\end{tabular}
	}
	= \rho_k\lrp{x_1,\dots,x_k}
		\prod_{i=1}^k \mu(\diff x_i),
\end{equation*}
see \citet[Section 4]{MoWa04}.
The functions $\rho_k$ describe the interaction among points in $\calX$ by quantifying co-occurrence of points at a set of locations.
% Considering $\mu$ as the reference measure, the $k$-correlation functions are defined as
% \begin{equation}
% \label{eq:k-correlation_function}
%   \Expe{ \sum_{
%     \substack{
%       (X_1,\dots,X_k) \\
%       X_1 \neq \dots \neq X_k} }
%     f(X_1,\dots,X_k)
%     }
%     = \int_{\bbX^k}
%       f(x_1,\dots,x_k) \rho_k(x_1,\dots,x_k)
%       \prod_{i=1}^k \mu(dx_i),
% \end{equation}

% for all bounded measurable functions $f:\bbX^k\to \bbC$ and the sum ranges over $k$-tuples of distinct points of $\calX$.

A point process $\calX$ on $(\bbX,\mu)$ parametrized by a kernel $K:\bbX\times \bbX\rightarrow \bbC$ is said to be \emph{determinantal}, denoted as $\calX\sim\DPP(K)$, if its $k$-correlation functions satisfy
  \begin{equation*}
  \label{eq:k-correlation_function_DPP}
    \rho_k(x_1,\dots,x_k)
      = \det \lrb{K(x_i, x_j)}_{i,j=1}^k,
    \quad \forall k\geq 1.
  \end{equation*}
	Most DPPs in ML correspond to the finite case where $\bbX = \lrcb{1,\dots,M}$ and the reference measure is $\mu=\sum_{i=1}^M \delta_i$ \citep{KuTa12}.
	In this context, the kernel function becomes an $M\times M$ matrix $\bfK$, and the correlation functions refer to inclusion probabilities.
DPPs are thus often defined in ML as $\calX\sim \DPP(\bfK)$ if
  \begin{equation}
  \label{eq:inclusion_proba_finite}
    \Proba{S \subset \calX} = \det {\bfK}_S,
      \quad\forall S\subset \bbX,
  \end{equation}
where ${\bfK}_S$ denotes the submatrix of $\bfK$ formed by the rows and columns indexed by $S$. \href{https://dppy.readthedocs.io/en/latest/finite_dpps/definition.html}{\textcolor{magenta}{Sometimes}}, an almost-equivalent definition is given using a related likelihood kernel $\mathbf{L}$ rather than the correlation kernel $\mathbf{K}$.
If we further assume that the kernel matrix $\bfK$ is Hermitian, then the existence and unicity of the DPP in \Eqref{eq:inclusion_proba_finite} is equivalent to the condition that the spectrum of $\bfK$ lies in $[0,1]$.
The result also holds for general Hermitian kernel functions $K$ with additional assumptions \cite[Theorem 3]{Sos00}.
We note that there are also DPPs with nonsymmetric kernels \citep{BoDiFu10}.

The main interest in DPPs in ML is that they can model diversity while being tractable.
To see how diversity shows, we simply observe that \Eqref{eq:inclusion_proba_finite} entails
\begin{equation*}
\label{eq:2point_correlation_function_inclusion_proba_finite_case}
  \Proba{\{i,j\} \subset \calX}
    = {\bfK}_{ii}{\bfK}_{jj}-{\bfK}_{ij}{\bfK}_{ji}
    = \Proba{\{i\} \subset \calX}
      \times \Proba{\{i\} \subset \calX}
        - {\bfK}_{ij}{\bfK}_{ji}.
\end{equation*}
If $\bfK$ is Hermitian, the quantity ${\bfK}_{ij}{\bfK}_{ji} = \vert{\bfK}_{ij}\vert^2$ encodes how less often $i$ and $j$ should co-occur, compared to independent sampling with the same marginals.
Most point processes that encode diversity are not tractable, in the sense that we do not have efficient algorithms to sample, marginalize, or compute normalization constants.
DPPs are amenable to these tasks \citep{KuTa12}.

% subsection definition (end)
