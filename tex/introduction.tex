% !TEX root = dppy_paper.tex 
\section{Introduction} % (fold)
\label{sec:introduction}

Determinantal point processes (DPPs) are distributions over configurations of points that encode diversity through a kernel function $K$.
They were introduced by \citet{Mac75} as models for beams of fermions, and they have since found applications in fields as diverse as probability \citep{Sos00, Kon05, HKPV06}, statistical physics \citep{PaBe11}, Monte Carlo methods \citep{BaHa16}, spatial statistics \citep{LaMoRu15}, and machine learning \citep[ML,][]{KuTa12}.

In ML, DPPs mainly serve to model diverse sets of items, as in recommendation \citep{KaDeKo16, GaPaKo16} or text summarization \citep{DuBa18}.
Consequently, MLers  use mostly finite DPPs, which are distributions over subsets of a finite \emph{ground set} of cardinality $M$, parametrized by an $M\times M$ kernel matrix $\bfK$.
Routine inference tasks such as normalization, marginalization, or sampling have complexity $\calO(M^3)$ \citep{KuTa12}.
For large $M$, $\calO(M^3)$ is a bottleneck, as for other kernel methods, see also \citet{TrBaAm18} for a survey on exact sampling.
Efficient approximate samplers have thus been developed, ranging from kernel approximation \citep{AKFT13} to MCMC samplers \citep{AnGhRe16, LiJeSr16c, GaBaVa17}.

In terms of software, the R library \textsf{spatstat}\ \citep{BaTu05}, a general-purpose toolbox on spatial point processes, includes sampling and learning of continuous DPPs with stationary kernels, as described by \citet{LaMoRu15}.
On the other hand, we propose \DPPy, a turnkey implementation of all known general algorithms to sample \emph{finite} DPPs.
We also provide a few algorithms for non-stationary continuous DPPs that are related to random projections and random covariance matrices, which can be of interest for MLers.

\DPPy\ is hosted on GitHub\footnoteref{fn:github}\!
+and we use \setcounter{footnote}{5}Travis\footnote{\url{https://travis-ci.com/guilgautier/DPPy}} for continuous integration.
Moreover, the project is supported by an extensive documentation\footnoteref{fn:docs} which provides the essential mathematical background and illustrates some key properties of DPPs through \DPPy\ objects and associated methods.
\DPPy\ thus also serves as a tutorial.
Along the paper, words in magenta point to the documentation.

% section introduction (end)